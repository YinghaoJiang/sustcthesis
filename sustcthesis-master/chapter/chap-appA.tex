\chapter{The TCP Throughput}
\label{chap:appA}
It is known to all networking professionals that Equation \ref{equ_tcpthroughput}
below is the throughput TCP can achieve at a steady state,
Equation \ref{equ_tcpthroughputmax} is roughly the maximum throughput TCP
can achieve.
\begin{equation}
 \label{equ_tcpthroughput}
  Throughput_{TCP}=\frac{min(RWND,CWND)}{RTT}
\end{equation}
\begin{equation}
 \label{equ_tcpthroughputmax}
  Throughput_{TCP_{max}}=\frac{RWND_{max}}{RTT}
\end{equation}
where RWND is the receiver window size, CWND is the congestion windows size (at the
sender side), and RTT is the round trip time.
Also, theoretically we have Equation 2.3 if considering the loss rate p.
\begin{equation}
 \label{equ_tcpthroughputmax}
  Throughput_{TCP} \approx \frac{1}{RTT}\sqrt{\frac{3}{2p}}
\end{equation}
Let’s consider transmitting a data chunk from node A to node B. When the distance
between A and B gets longer, generally the RTT is larger for a certain type of transmission
medium. In addition, p is also likely to be larger, as there are likely to be more routers
along the path from A to B, and their performance, buffer window sizes, and current
congestion status may vary. Thus, any single node in between can be a bottleneck.
In addition to the window sizes, the packet loss rate p, and the RTT, there are other
factors that affect TCP’s performance: the slow-start and the 3-way handshake behaviour.
