\chapter{Preface}
\label{chap:chap-preface}
This thesis is intended to summarize the learning outcomes and some works done by me during the undergraduate period in south university of science and technology of china (SUSTC). In this thesis, I choose two topics, one is related to algebraic number theory and the other one is something about function fields. The central idea or tool is the Galois theory, which connects the two part for this paper. For the number fields part, I have rearrange many references and books in order to give a simple and clear skeleton. For the function fields, we prove an explicit theorem, which is useful for studying the discrete (finite) subgroups of the automorphism group of a function field.  

Before the main contents of the thesis, I would like to take some words for my process on studying mathematics during the time in SUSTC. Although there is no mathematics department for our newly established university, I am grateful that I have learn many mathematics, especially abstract algebra, algebraic number theory and algebraic geometry from Prof. Xianke Zhang, and  Prof. Jietai Yu in the University of Hongkong etc.

As a student who loves mathematics very well, I have met so many kindly and responsible mathematicians during my 3.5-years' college life. During the first two years, as the 
representative of mathematics lessons in our inaugural class across all majors, I have ``obligation" to get good grades on mathematics lessons in order to set a good example. I learned Calculus and Analysis from Prof. Xuefeng Wang in mathematics department of Tulane University, Prof. Jinzhong Zhang in Guangzhou University, Prof. Zhongkan Liu in SUSTC and Beihang University, who is my first supervisor. We had frequent discussions on some topics on analysis as well as many other topics like elementary geometry and ordinary differential equation etc. 

After entering financial mathematics department, I also learned ordinary differential equation, probability, stochastic process, numerical methods, as well as some statistics from the professors in the department. Among of them, Prof. Anyue Chen, Prof. Jingzhi Li, Prof. Xuejun Jiang, Prof. Huaiqing Wang, Prof. Dejun Xie and Prof. Bianxia Sun gave me so many supports. Especially, Prof. Li, as my graduation designation supervisor, encouraged me a lot on studying algebra, and helped me to know some good mathematician friends of him.

Prof. Xianke Zhang, supervisor of me (after Prof. Liu) and an academic supervisor of this thesis, has guided me into the world of abstract algebra, a field which is quite abstruse for freshman. However, I found appealing thing inside the abstract definitions and complex relations one day. After that, besides the formal classes, we held several seminars on algebraic number theory and commutative algebra etc under the guidances of Prof. Zhang. 

Prof. Jietai Yu who has inspired me on Galois theory and algebraic geometry, has given me many guidances on research area as well as daily life. He gave us Galois courses as well as seminar on algebraic geometry. Among that time, I had my first opportunity to give a presentation in the formal seminar. I still remember the topics for that seminar is the finite subgroups of $\operatorname{PGL}(2,\mathbb{C})$ (we will show the relevant topic in chapter \ref{chap:chap-eight}). What's more, Prof. Yu found more opportunities for me to learn more mathematics as well as to further my study.

During those experiences outside the campus, I have learnt more recent works on affine algebraic geometry. The happiest things for me during those academic activities are finding many good friends, like Swapnil Lokhande, Sagar Kolte, Shameek Paul, Shihong Ma, Ju Huang, Haifeng Tian etc. They really gave me so much encouragement and concern. Also, I also met many good mathematician, like Prof. Alexey Belov, Prof. Leniod Makar-Limanov, Prof. Wenhua Zhao, Prof. Fang Li, Prof. Xiankun Du and so on. They gave me some instructions on some problems and encouraged me on studying mathematics.

Last but not least, I've come to realize that the ability of self-study is the most important good properties for a university student. I have paid much time to learn something outside the courses, to find some interesting topics, to sort out learning knowledge. During these years, I wrote a personal mathematics blog to record those things.


\vskip 28pt

\begin{flushright}

Wenchao ZHANG

September, 2014 at SUSTC

\end{flushright}






     